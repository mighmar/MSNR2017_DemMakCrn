\documentclass[12pt]{article}
\usepackage[utf8]{inputenc}
\usepackage[T1]{fontenc}
\usepackage[table]{xcolor}
\usepackage{pgfgantt}

\begin{document}

\title{Apstraktna Interpretacija}
\author{Ozren Demonja, Stefan Maksimović, Marko Crnobrnja}
\date{10.~april 2017.}
\maketitle

Plan izrade seminarskog rada iz predmeta Metodologija naučnog i stručnog rada, na temu apstraktna interpretacija. \\


\setlength{\arrayrulewidth}{0.5mm}
\setlength{\tabcolsep}{12pt}
\renewcommand{\arraystretch}{1.0}


{\rowcolors{3}{gray!25}{white}
\begin{tabular}{ |p{3cm}|p{2cm}|p{4cm}|  }
\hline
\multicolumn{3}{|c|}{Plan izrade} \\
\hline
Naziv &Rok &Opis posla \\
\hline
Istraživanje &16. Mart &Istraživanje teme, pronalaženje potrebne  literature, usklađivanje i podešavanje \LaTeX \\
Podela posla &17. Mart &Podela poslova i utvrđivanje potrebnog broja sati za rad i težine svakog posla ponaosob\\
Prva verzija &23. Mart &Napisan kostur rada, završeno minimum 40\% celokupnog posla \\
Druga verzija &30. Mart &Rad je napisan u celosti \\
Lektorisanje &6. Mart &Lektura rada, popisane greške \\
Treća verzija &15. Mart &Ispravljanje uočenih grešaka, završno čitanje \\
Preliminarna verzija &2. Maj &Ispravljanje grešaka u skladu sa recenzijama, odgovori na recenzije     \\
Finalna verzija &12. Maj &Predavanje finalne verzije     \\

\hline
\end{tabular}
}
\\
\\

Fakultetske obaveze(projekti, testovi...), moguća odsutnost članova tima zbog poslovnih putovanja i kolokvijumska nedelja mogu uticati na manja probijanja rokova.
Svakodnevna komunikacija članova tima tokom celog projekta, stalno praćenje napretka, konstantan tempo rada radi smanjenja mogućnosti prekoračenja rokova, po potrebi kosultovanje sa predmetnim nastavnikom.
\newline

\begin{ganttchart}[x unit=1.2cm]{1}{8}
  \gantttitle{Mart}{4} 
  \gantttitle{April}{2} 
  \gantttitle{Maj}{2} \\
  \gantttitlelist{16, 17, 23, 30, 6, 15, 2, 12}{1} \\
  \ganttbar{Istraživanje}{1}{1} \\
  \ganttbar{Podela posla}{2}{2} \ganttnewline
  \ganttbar{Prva verzija}{3}{4} \\  
  \ganttbar{Druga verzija}{4}{5} \\ 
  \ganttbar{Lektorisanje}{5}{5} \\
  \ganttbar{Treća verzija}{6}{6} \\
  \ganttbar{Preliminarna verzija}{7}{7} \\ 
  \ganttbar{Finalna verzija}{8}{8} \\ 
  \ganttlink{elem0}{elem1}
  \ganttlink{elem1}{elem2} 
  \ganttlink{elem2}{elem3}
  \ganttlink{elem3}{elem4}
  \ganttlink{elem4}{elem5}
  \ganttlink{elem5}{elem6}
  \ganttlink{elem6}{elem7}
\end{ganttchart}
\center gant dijagram plana izrade
   
\end{document}

\end{document}