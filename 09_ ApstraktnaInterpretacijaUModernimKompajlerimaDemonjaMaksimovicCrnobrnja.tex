% !TEX encoding = UTF-8 Unicode

\documentclass[a4paper]{article}

\usepackage{color}
\usepackage{url}
\usepackage[utf8]{inputenc} % make weird characters work
\usepackage{graphicx}
\usepackage{amsmath}
\usepackage{amssymb}
\usepackage{amsthm}
\usepackage{listings}
\usepackage{verbatim}
\usepackage{float}

\lstdefinestyle{stil} {
	numberstyle=\tiny\color[rgb]{0.6, 0.6, 0.6},
	numbers=left
}


\lstset{style=stil}

\usepackage[english,serbian]{babel}

\usepackage[unicode]{hyperref}
\hypersetup{colorlinks,citecolor=green,filecolor=green,linkcolor=blue,urlcolor=blue}

\newtheorem{primer}{Primer}[section]
\newtheorem{definicija}{Definicija}
\newtheorem{teorema}{Teorema}

\usepackage{mathptmx}

\usepackage{graphicx}
\makeatletter
\providecommand{\bigsqcap}{%
  \mathop{%
    \mathpalette\@updown\bigsqcup
  }%
}
\newcommand*{\@updown}[2]{%
  \rotatebox[origin=c]{180}{$\m@th#1#2$}%
}
\makeatother

\begin{document}

\title{Apstraktna interpretacija\\ \small{Seminarski rad u okviru kursa\\Metodologija stručnog i naučnog rada\\ Matematički fakultet}}

\author{Ozren Demonja, Stefan Maksimović, Marko Crnobrnja\\ \{mi12319, mi12078, mi12024\}@alas.matf.bg.ac.rs}
\date{1.~april 2017.}
\maketitle

\abstract{U ovom tekstu predstavljena je metoda apstraktne interpretacije programskog koda, objašnjeni uslovi njenog nastajanja i navedene njene primene u savremenom računarstvu za optimizaciju i verifikaciju softvera. Izložena je matematička formalizacija koja ukazuje na valjanost upotrebe ove metode za pouzdanu verifikaciju softvera kao i celo poglavlje u kome je detaljno razmotren primer dva C++ koda.}

\tableofcontents

\newpage

\section{Uvod}
\label{sec:uvod}
%U protekle dve decenije se dosta toga promenilo u pogledu performansi računara. Današnji kućni računari su po snazi jači nego najmoćniji superračunari iz 70-ih. U međuvremenu, kroz paralelizovanje i inovacije u hijerarhiji memorije superračunari sada postižu 10 do 100 teraflopa(engl. floating point operations per second) \cite{allen2001optimizing}. \\
%
%Glavne krivce za ovakav porast brzine računara pronalazimo kroz dva aspekta. Prvi, osnovna tehnologija prema kojoj se računari konstruišu je doživela izuzetan napredak koji počiva na predviđanjima Murovog zakona (engl. Moore's law) \cite{Schaller1997MooresLaw}. Drugi aspekt je paralelizam u nekoj svojoj formi \cite{allen2001optimizing}.  \\	

%Ova poboljšanja u snazi nisu došla bez problema. Kako je arhitektura postajala kompleksnija ne bi li mogla pratiti eksponencijalnu brzinu Murovog zakona, postajalo je sve teže i teže programirati. Većina vrhunskih programera je postala svesna potrebe da eksplicitno upravlja memorijom. U naporu da se poboljšaju performanse pojedinačnih procesa, programeri su učili kako da ručno transformišu njihov kod tako da se efikasnije izvrši planiranje instrukcija na višeprocesorskom sistemu \cite{allen2001optimizing}. \\

U protekle dve decenije se dosta toga promenilo u pogledu performansi računara, arhitektura je postala znatno kompleksnija ne bi li mogla pratiti eksponencijalnu brzinu Murovog zakona i trendove u višeprocesorskom računanju. Zbog toga je u  današnje vreme značajni deo koda u većini modernih kompajlera je posvećen optimizaciji generisanog koda. Često se dešava da je ponašanje pri izvršavanju optimizovanog koda nesaglasno sa pre-optimizovanim ponašanjem koda, drugim rečima optimizacija je uticala kako na semantiku programa tako i na pragmatiku. Ovaj problem se često dešava zbog nedovoljne strogosti koja je bila primenjena na ispravnost dokaza optimizacije. Za programske jezike sa definisanom matematičkom semantikom postoji rastući skup alata koji obezbeđuju osnovu za semantički korektnu transformaciju, jedan od tih alata je i apstraktna interpretacija \cite{AbramskyHankin}. \\



\section{Apstraktna interpretacija}
\label{sec:Apstraktna interpretacija}

Kao što se vidi iz prethodnog poglavlja apstraktna interpretacija je tehnika za automatsku statičku analizu. Sastoji se od zamene preciznih elemenata programa sa manje detaljnim apstrakcijama. Apstrakcija dovodi do gubitka sigurnih informacija, što dovodi do nemogućnosti dovođenja zaključaka za sve programe. Apstraktna interpretacija omogućava da otkrijemo greške nastale tokom izvršavanja programa, kao što su deljenje sa 0, prekoračenje, itd, a takođe otkriva korišćenje zajedničkih promenljivih i mrtvih petlji \cite{AbramskyHankin}. \\ 

Glavna prednost alata koji koriste apstraktnu interpretaciju je da se test obavlja bez ikakve pripreme, baziran na kodu projekta. Ako se uporedi sa troškovima jediničnog testiranja, to predstavlja značajan argument \cite{AbramskyHankin}. \\


\subsection{Problem koji se rešava}
\label{subsec:problem1}
Da bi se lakše shvatio problem prvo ćemo pokazati tri neprogramerska primera apstraktne interpretacije. Ovi primeri će služiti za uspostavljanje principa pristupa. \\

Pretpostavimo da želimo da putujemo negde. Jedna od odluka koju moramo napraviti je da li želimo da hodamo, vozimo se ili letimo. Umesto da ovu odluku sprovodimo metodom pokušaja i greške, mi ćemo koristiti osobinu putovanja, udaljenost (koju možemo izmeriti na mapi) da odlučimo koji je najbolji način transporta. Mapa je apstraktna reprezentacija putovanja i merenjem rastojanja mi apstrahujemo sam proces putovanja \cite{AbramskyHankin}. \\

Slično ovom, može nam biti zadato da odredimo za neki broj da li je paran ili neparan. Sve što trebamo učiniti u tom slučaju jest videti da li je najmanje značajna cifra broja parna ili neparna - zadatak koji
zahteva manje računarskog napora nego deliti celi broj za dva (osim ako je jednocifren broj). \\


Treći primer, formalniji, gradi se korišćenjem pravila znaka. Određujemo znak rezultata množenja. Ako se pitamo koji je znak

336 * (-398)  

mi odmah znamo da je rezultat negativan. Bez izvođenja operacije množenja mi određujemo znak na osnovu pravila znaka. Znamo da će množenje pozitivnog i negativnog broja uvek proizvesti za rezultat negativan broj. Ovaj treći primer je malo bliži apstraktnoj interpretaciji kod programiranja tako da ćemo malo dublje zaći u njega \cite{AbramskyHankin}. \\

Da bismo razumeli apstraktnu interpretaciju moramo da prebacimo zadatak u sledeću formu:

\begin{equation}
  a_{+} \times a_{-}
\end{equation}

gde $\times$ predstavlja pravilo znaka pri množenju, $a_{+}$ pozitivan, a $a_{-}$ negativan broj. Zatim u ovakvom zapisu izvodimo sledeće jednostavije izraze:  \\

\begin{multline}
  	0 \times a_{+} = 0 \times a_{-} = a_{+} \times 0 = a_{-} \times 0 = 0 \\
  	a_{+} \times a_{+} = a_{-} \times a_{-} = a_{+} \\
  	a_{+} \times a_{-} = a_{-} \times a_{+} = a_{-} \\
\end{multline}

Do sada nismo razmatrali korektnost interpretacije, ali treba da bude jasno da možemo dobiti potpuno tačne odgovore u oba primera. Ova situacija postaje mnogo nejasnija ako umesto množenja stavimo sabiranje. Prvih nekoliko redova ne pred\-stavljaju neki problem \\

\begin{multline} \\
	0 \pm a_{+} = a_{+} \pm 0 = a_{+,0} \\ 
	0 \pm a_{-} = a_{-} \pm 0 = a_{-,0} \\ 
	a_{+} \pm a_{+} = a_{+}  \\ 
	a_{-} \pm a_{-} = a_{-}  \\ 
\end{multline}


Ali ostatak je problematičan:

\begin{multline}	\\
	a_{+} \pm a_{-} = ?? \\ 
	a_{-} \pm a_{+} = ?? \\ 
\end{multline}

Ako bi stavili znak (0, +, -), a da ne znamo vrednosti u nekim slu\-čajevima bi pogrešili jer odgovor zavisi od vrednosti na koje se primenjuje. Kako možemo da okarakterišemo pravi izbor za ?? ? Da bi mogli to da uradimo moramo da znamo koji znak u apstraktnom izračunavanju predstavlja:

\begin{multline}	\\
	a_{0} = \{0\}				\\    
	a_{+} = \{n \mid n > 0\}		\\	
	a_{-} = \{n \mid n < 0\}		\\
\end{multline}

Onda je apstraktna kalkulacija tačna ako je pravi odgovor član skupa koji apstraktni odgovor predstavlja. Ako je ovo slučaj kaže se da je apstraktna interpretacija sigurna. Ako koristimo $a$ da predstavimo cele brojeve, dobijamo sigurnu verziju sabiranja dodavanjem pravila:

\begin{multline}	
	s \pm a = a \pm s = a \pm a = a \quad \text{gde je s} \in \{0, -, +\}
\end{multline}
\cite{AbramskyHankin}

\subsection{Korišćenje u računarstvu}
\label{subsec:problem1}
Kako je apstraktna interpretacija korisna u računarstvu? Mnogi tradicionalni optimizatori koji su zasnovani na toku upravljanja (eng. Control flow) i na analizi toka podataka (eng. Data-flow analysis) se uklapaju u okvir apstraktne interpretacije. Neke posebne analize koje su značajne u deklarativnim jezicima su:

\begin{itemize}
\item Stroga analiza (eng. Strictness analysis):
Analiza koja omogućava optimizaciju lenjih  funkcionalnih programa identifikujući parametre koji mogu biti prosleđeni po vrednosti tako da se izbegne potreba za pravljenjem zatvorenja (eng. closure) i otvara se mogućnost paralelne evaluacije. 

\item Analiza menjanja u mestu (eng. In-place update analysis):
Ova analiza nam omogućava da odredimo tačke u programu na kojima je sigurno da se uništi objekat jer ni jedan pokazivač ne pokazuje na njega. Rezultate u ovoj oblasti je doneo Hudak. Značajan rezultat je, po prvi put, funkcionalna verzija kviksort (engl. quicksort) algoritma koja može da se pokrene u linearnom prostoru. \cite{Girard1987}

\item Analiza relevantnih klauza (eng. Relevant clause analysis):
U mnogim prototipovima 5. generacije arhitekture programi mogu da naprave nelokalni pristup definicijama funkcija. Ovo povlači da postoji komunikacija povezana sa izvršavanjem programa. Korišćenjem analize delova postaje moguće identifikovati delove definicije funkcije koji su relevantni za naš program i tako smanjiti troškove.  

\item Analiza moda (eng. Mode analysis):
Značajno povećanje performansi može se postići u Prologu ako se zna kako se logičke promenjljive koriste u relaciji (kao ulazne, izlazne ili oba).
Kada je deklarativna zajednica postala svesna apstraktne interpretacije, nove aplikacije su otkrivene. 

\end{itemize}
Optimizacije zasnovane na apstraktnoj interpretaciji su verovatno tačne. Ako ovo prebacimo u gornje primere to bi bilo:

\begin{itemize}
\item Stroga analiza:
Ako stroga analiza utvrdi da je funkcija stroga u argumentima onda to ona definitivno i jeste, ali analiza neće uspeti da detektuje neke parametre koji mogu biti prosleđeni po vrednosti. 

\item Analiza menjanja u mestu:
Ako analiza menjanja u mestu ukaže da možemo destruktivno da ažuriramo podatke onda i možemo ali ćemo kopirati neke objekte koji su mogli biti uništeni. \cite{Girard1987}

\item Analiza relevantnih klauza:
Analiza relevantnih klauza će nas terati da komuniciramo sa nadskupom koda koji je u stvari samo potreban za neke posebne aplikacije.  

\item Analiza moda:
Analiza moda nekad neće uspeti da detektuje logičke promenjljive koje se isključivo koriste kao ulazno-izlazne promenjljive. 

\end{itemize}


\section{Formalizacija} 
\label{sec:Formalizacija}
Označimo izvršno stanje programa u datom trenutku, pod čime se podrazumeva vrednost promenljivih kao i mesto u kodu do koga se došlo, odnosno na koje pokazuje programski brojač, sa $v \in V$ gde je $V$ skup svih takvih stanja. Tada možemo primetiti binarnu relaciju prelaska stanja $v_{0} \rightsquigarrow v_{1}$ koja predstavlja da stanje $v_{1}$ može uslediti za stanjem $v_{0}$. \footnote{Ovo poglavlje se primarno oslanja na \cite{salcianu}, gde se mogu naći dokazi tvrdnji koji su ovde izostavljeni radi sažetosti.} \\

Bitno je napomenuti da se ova relacija ne može zameniti funkcijom koja bi slikala jedno stanje u iduće, jer prelazak može zavisiti od okolnosti koje nisu definisane unutar programa, poput učitavanja podataka ili redosleda izvršavanja instrukcija u slučajevima kada program ima više niti, tako da može postojati više različitih stanja u koje jedno stanje prelazi. Posebno su nam zanimljiva stanja $$... \rightsquigarrow v_{n} \rightsquigarrow v_{n}$$ koja odgovaraju zaustavljanjima programa. \\
 
Budući da je u opštem slučaju jedini način da odredimo $v$ da izvršimo sam program, uopštićemo problem uvođenjem pojma prostora svojstava $L$. Njegovi elementi $l \in L$, koje nazivamo apstraktnim stanjima, obuhvataće svojstva koja stanja u koja program dospeva u datom trenutku imaju. Potrebno je naglasiti da dato apstraktno stanje ne predstavlja svojstva jednog konkretnog stanja, već više konkretnih stanja te da pojedinačne promenljive apstraktnih stanja uzimaju vrednosti iz partitivnog domena odnosno skupa podskupova domena promenljivih u konkretnim stanjima.\cite{denotationalsemantics} \\

Nad prostorom svojstava već možemo definisati funkciju $f_{L}:L\rightarrow L$ koja slika apstraktno stanje u ono koje mu sledi.

Pokazaće se korisno definisati dodatnu strukturu nad $L$: \\
\begin{definicija}
Neka je nad $L$ definisana relacija poretka, odnosno relacija $\sqsubseteq$ takva da za sve $a, b, c \in L$ važi
\begin{enumerate}
\item $a \sqsubseteq a$ (Refleksivnost)
\item ako su $a \sqsubseteq b$ i $b \sqsubseteq a$ tada $a = b$ (Antisimetričnost)
\item ako su $a \sqsubseteq b$ i $b \sqsubseteq c$ tada $a \sqsubseteq c$ (Tranzitivnost)
\end{enumerate}
Ukoliko za svaki podskup $L\prime \subseteq L$ postoji najmanja gornja granica $\bigsqcup L^{\prime}$ i najveća donja granica $\bigsqcap L^{\prime}$ tada se $L$ naziva \emph{potpunom mrežom}. \cite{algebra}
\end{definicija} 

\begin{figure}
\begin{center}
\includegraphics[scale=0.5]{Rho.png}
\end{center}
\caption{Odnos između apstraktnih i konkretnih stanja}
\label{fig:Rho}
\end{figure}

Relacija poretka koju uvodimo nad prostorom svojstava je takva da su veći elementi opštiji od manjih, odnosno da predstavljaju slabije tvrđenje o stanju programa. Tada $\bigsqcup L^{\prime}$ predstavlja disjunkciju apstraktnih stanja u $L^{\prime}$ odnosno stanje u kome važe bilo koja od datih svojstava, dok $\bigsqcap L^{\prime}$ označava stanje u kome sva svojstva važe. Bitne vrednosti su takođe i $\bigsqcup L = \top$ i $\bigsqcap L = \bot$.

Sada želimo dovesti u vezu konkretna stanja programa sa apstraktnim stanjima koja ih modeliraju putem relacije $\rho \subseteq V \times L$ kao što je prikazano na slici \ref{fig:Rho} \footnote{slika je preuzeta sa izmenama iz \cite{salcianu}}. Zahtevamo sledeće od ove relacije:
\begin{enumerate}
\item $\forall v,\, l_{1},\, l_{2},\, (v\: \rho \: l_{1}) \vee (l_{1} \sqsubseteq l_{2}) \Rightarrow (v\: \rho \: l_{2})$
\item $\forall v,\, L^{\prime} \subseteq L, (\forall l \in L^{\prime}, 	(v\: \rho \: l)) \Rightarrow v\: \rho \: (\bigsqcap L^{\prime})$
\end{enumerate}

Ovakvu relaciju nazivamo \emph{relacijom ispravnosti}. Da bi dokazali njenu valjanost u konkretnom slučaju, dovoljno je dokazati je za početno stanje izvršavanja i pokazati da se valjanost čuva pri svakom prelasku u iduće stanje. 


\subsection{Fiksne tačke}
Ukoliko bismo želeli saznati svojstva programa $l$ u nekoj tački izvršavanja, najdirektniji i najprecizniji način bi bio da izračunamo sva apstraktna stanja $l_{i} \in W(l)$ dobijena duž putanja izvršavanja koja vode do te tačke od početnog stanja $l_{0}$ i zatim nađemo $\bigsqcup W(l) = l$. 

Nažalost, u praksi je takav račun nemoguć ili makar veoma zahtevan. Umesto toga, računaju se fiksne tačke funkcije $x = f_{L}(x)$ koje takođe čine potpunu mrežu pod uslovom da je $f_{L}$ \emph{monotona} \cite{tarski}, odnosno da važi $$\forall l_{1},\, l_{2},\, l_{1} \sqsubseteq l_{2} \, \Rightarrow \, f_{L}(l_1) \sqsubseteq f_{L}(l_2)$$ 

 Ako je uz to funkcija i \emph{neprekidna}, $\bigsqcup f_{L}[L^{\prime}] = f_{L}[\bigsqcup L^{\prime}]$, tada se najmanja fiksna tačka može izračunati kao $$\bigsqcup \{ \: f^{n}_{L}(\bot)\: \}_{n \in \mathbb{N}} \quad \text{gde je} \quad f^{0}_{L}(\bot) = \bot \quad \text{i} \quad f^{n}_{L}(\bot) = f_{L}(f^{n-1}_{L}(\bot)) \quad \cite{baranga}$$

Ipak, i ovom slučaju niz $f^{n}_{L} ( \bot )$ može previše sporo konvergirati, zbog čega uvodimo još jedan, grublji, prostor svojstava $M$ koji će služiti kao apstrakcija za $L$. Da bi objasnili odnos između ova dva prostora, moramo uvesti koncept galoaove veze:
\begin{definicija}
Neka su $(A, \geqslant)$ i $(B, \geqslant)$ parcijalno uređeni skupovi a $F : A \rightarrow B$ i $G : B \rightarrow A$ monotone funkcije. Tada je $\langle A, 	F, G, B \rangle$ \emph{galoaova veza} ukoliko važi 
\begin{enumerate}
\item $\forall a \in A, \, a \leqslant G (F (a))$ 
\item $\forall b \in B, \, b \geqslant F (G (b))$
\end{enumerate}
\end{definicija} 

\begin{teorema}
Ako između $L$ i $M$ postoji galoaova veza $\langle L, \alpha, \gamma, B \rangle$ tada je $\rho^{\prime} \subseteq M \times V$, takva da $$m\: \rho^{\prime}\: v \iff \gamma (m)\: \rho \: v$$ takođe relacija ispravnosti.
\end{teorema}

Druga tehnika je korišćenje operatora proširenja $\nabla : L \times L \rightarrow L$ takvog da je $x, y \sqsubseteq x \nabla y$ za sve $x, y$ i pomoću koga se za bilo koji rastući niz $(y_{n})_{n}$ može napraviti niz $$(x^{\prime}_{n})_{n} \quad \text{gde je} \quad x^{\prime}_{0} = y_{0} \quad \text{i} \quad x^{\prime}_{n} = x^{\prime}_{n-1} \nabla y_{n} $$ takav da konvergira u konačnom broju koraka. \\

Najčešće za funkciju prelaska $f_{L}$ pravimo niz $(f^{n}_{\nabla})_{n}$ takav da:
$$
f^{n}_{\nabla} = 
\begin{cases}
\bot,            								  & 	\text{za} \quad n = 0 \\
f^{n-1}_{\nabla} 							      & \text{za} \quad n > 0 \quad \text{i} \quad f_{L}(f^{n-1}_{\nabla}) \sqsubseteq f^{n-1}_{\nabla} \\
f^{n-1}_{\nabla} \nabla f_{L}(f^{n-1}_{\nabla})  & \text{inače}
\end{cases}
$$

Ovime efektivno ubrzavamo nizove koji rastu a zaustavljamo ih u suprotnom, time se sprečava zaglavljivanje prilikom analizi petlji i drugih cikličnih tokova upravljanja. Za limes ovog niza ispostavlja se da je veći od najmanje fiksne tačke, te da ga dobro aproksimira. 

\subsection{Pro upotrebe operatora proširenja}
\label{subsec:widening}

Razmotrimo sada 

\lstinputlisting[language=C, caption=Primer koda 2]{snippets/ex3.cpp}


\section{Primeri}
\label{sec:Primeri}
Objasnićemo apstraktnu interpretaciju na primeru \emph{propagacije konstanti}. 
Cilj nam je da otkrijemo u svakoj tački funkcije da li bilo koja od promenljivih koja se koristi u toj tački ima konstantnu vrednost, tj. da li ima istu vrednost nezavisno od ulaznih parametara funkcije i nezavisno od toga koji deo koda je izvršen u toj funkciji. 
Prevodioci koriste ovaj tip analize za optimizaciju propagacije konstanti, što znači menjanje konstantnih promenljivih konstantama. 
Ovo je primer C++ koda koji ćemo analizirati, sa komentarima koji ukazuju na konstantne promenljive.
\lstinputlisting[language=C++, caption=Primer koda]{snippets/ex1.cpp}

\subsection{Grafovi kontrole toka}
\label{subsec:cfgs}

\begin{figure}[H]
\begin{center}
\includegraphics[scale=0.5]{Treehydra-cfg.png}
\end{center}
\caption{Primer grafa kontole toka}
\label{fig:graf}
\end{figure}

Apstraktna interpretacija se obavlja nad dijagramom koji predstavlja funkciju koju ispitujemo, i zove se \emph{graf kontrole toka (eng. control flow graph, CFG}). Na slici \ref{fig:graf} je prikazan graf funkcije koju ćemo ispitivati. \\
Neke napomene:
\begin{itemize}
\item Svi mogući prelazi su prikazani kao grane, tj. veze između čvorova koji sadrže kod
\item Naredbe imaju tačno jednu operaciju i najviše jednu dodelu. Privremene promenljive se dodaju po potrebi.
\end{itemize}

Terminologije:
\begin{itemize}
\item Svaki čvor se zove \textbf{osnovni blok} \emph{(eng. basic block, \textbf{BB})}. Osnovni blok se definiše tako što ima samo jednu tačku ulaza, i jednu tačku izlaza, što će reći da nema grananja unutar osnovnih blokova.
\item Naredbe ćemo zvati \emph{instrukcije}, iako one uopšteno mogu imati različite nazive u zavisnosti od toga koliko operanada primaju.
\item \textbf{Tačka u programu} je zamišljena tačka pre ili posle svake instrukcije. Funkcija ima dobro definisano stanje u svakoj tački, tako da će se naša analiza programa uvek referisati na ove tačke.
\end{itemize}


\subsection{Konkretna interpretacija}
\label{subsec:concreteimpr}
Kako bismo objasnili apstraktnu interpretaciju, počećemo prvo sa primerom konkretne interpretacije.
Kasnije ćemo se nadograditi na ovaj primer kako bismo objasnili apstraktnu interpretaciju. \\
Mogli bismo početi tako što bismo zvali funkciju za različite ulaze, i potom gledali koje su sve promenljive konstantne
kroz sve te pozive. Počnimo tako što ćemo pokrenuti program za ulaze \texttt{a=0, b=7}:
\verbatiminput{snippets/ex2_1.txt}
Dakle, \texttt{k = 2} pre nego što se koristi u naredbi \texttt{t1 := k + m}. Možemo pokrenuti funkciju za ostale ulaze i dobili bismo isti rezultat.
Međutim, ovakav način testiranja nam ne može potvrditi da je \texttt{k = 2} za sve moguće ulaze. (Doduše može, ali samo ako bismo proverili za svaki od $2^{64}$ ulaza.)


\subsection{Približavanje apstraktnoj interpretaciji}
\label{subsec:approachingabsint}
Ako pogledamo prethodnu funkciju, možemo primetiti da postoje samo dva bitna slučaja: \texttt{a == 0}, \texttt{a != 0}, dok
\texttt{b} nije bitno. 
Pokrenimo dva testa: jedan sa ulazom \texttt{a = 0, b = ?}, a drugi sa ulazom \texttt{a = NN, b = ?}, gde \textbf{NN} označava ne-nula vrednost, dok \textbf{?} označava bilo koju vrednost. Počnimo sa \texttt{a = 0, b = ?}:
\verbatiminput{snippets/ex2_2.txt}
Ovo izgleda poprilično isto kao i konkretan primer, samo što su sada neke vrednosti apstrahovane, \texttt{NN} i \texttt{?},
koje predstavljaju skupove konkretnih vrednosti.
Takođe moramo da znamo šta operatori rade nad apstraktnim vrednostima. Na primer, u poslednjem koraku,
\texttt{ret := t1 + n} postaje \texttt{ret := 2 + ?}. Kako bismo saznali šta ovo znači, posmatramo skupove konkretnih vrednosti: Ako \texttt{?} može biti bilo koja vrednost, onda i \texttt{2 + ?} takođe može uzeti bilo koju vrednost, tako da \texttt{2 + ? $\longrightarrow$ ?}.
Preostali slučaj testira \texttt{a = NN, b = ?}:
\verbatiminput{snippets/ex2_3.txt}
Sada smo testirali za svaki mogući ulaz, kao i svaku granu koda funkcije. Ovo je dokaz da \texttt{k = 2} je uvek tačno
pre nego dođemo do \texttt{t1 := k + m}.\\
Procedura koju smo upravo ispratili daje određen uvid kako bismo krenuli u proces apstraktne interpretacije, ali
nismo generalizovali samu proceduru. Tačno smo znali koje apstraktne vrednosti da koristimo za test slučajeve, i to smo mogli
samo zato što smo imali kao primer jednostavnu funkciju. Ova metoda neće biti primenjiva na komplikovane funkcije, i nije
automatizovana.\\
Drugi problem je što smo posmatrali svaku granu funkcije posebno. Funckija sa \texttt{k} iskaza može imati i do
\texttt{$2^k$} grana, dok funkcija sa petljama ih može imati i beskonačno, i ovo nam onemogućava da imamo kompletnu
pokrivenost.

\subsection{Apstraktna interpretacija kroz primer}
\label{subsec:absintex}
Jedan od problema sa gornjim pristupom apstraktnoj interpretaciji je bio što nismo znali kako da odaberemo skupove
apstraktnih vrednosti koje ćemo koristiti kao ulaz za test primere. Pokušajmo da sprovedemo jedan test gde nećemo birati
takve skupove, dakle pokušajmo sa sledećim ulazom: \texttt{a = ?, b = ?}:
\verbatiminput{snippets/ex2_4.txt}
Šta sada? Nemamo informaciju o tome šta je \texttt{a}, tako da ne znamo kojom granom treba da idemo. Odabraćemo obe. 
Prvo za potvrdnu granu:
\verbatiminput{snippets/ex2_5.txt}
Potom za negativni slučaj:
\verbatiminput{snippets/ex2_6.txt}
U ovoj tački, dva izvršna toka se spajaju. Mogli bismo da nastavimo da ih testiramo ponaosob, ali znamo da će to dovesti
do eksplozije u uopštenom slučaju, tako da ćemo izvršiti spajanje stanja. Potrebno nam je jedno stanje koje pokriva obe
grane:
\verbatiminput{snippets/ex2_7.txt}
Ovo stanje možemo dobiti tako što ćemo spajati promenljivu po promenljivu. Na primer, \texttt{k} je 2 u jednom i u drugom
stanju, tako da je \texttt{k = 2} u rezultujućem stanju. Za \texttt{m}, ono može biti bilo šta u prvom stanju, tako da iako
je ono 0 u drugom stanju, može uzeti bilo koju vrednost u rezultujućem stanju. Kao rezultat dobijamo:  
\verbatiminput{snippets/ex2_8.txt}
Možemo nastaviti izvršavanje u jednom toku:
\verbatiminput{snippets/ex2_9.txt}
Gde dobijamo odgovor koji smo želeli, \texttt{k = 2}. Osnovne ideje su bile:
\begin{itemize}
\item Proći kroz funkciju koristeći apstraktne vrednosti kao ulaz
\item Apstraktna vrednost predstavlja skup konkretnih vrednosti
\item Kod kontrole toka gde imamo grananje, krenimo put obe grane
\item Gde imamo spajanje, spajamo izlaz iz obe grane
\end{itemize}
\cite{MozWiki}



\section{Zaključak}
\label{sec:zakljucak}
U ovom radu smo pokušali da predstavimo tehniku apstraktne interpretacije na način koji će biti razumljiv onima koji nisu imali pređašnjeg kontakta sa teorijom verifikacije programa ili semantičke analize. Prišli smo temi iz neformalnog, tehničkog i formalno-matematičkog ugla. Nažalost, ovakva nam podela nije omogućila da zađemo dublje u materiju. Nadamo se ipak da je zahvaljujući njoj čitalac našao u skladu sa svojim sklonostima nešto što bi ga zainteresovalo za dalje proučavanje ove oblasti, jer je apstraktna interpretacija nesumnjivo korisna i visoko prilagodljiva metoda koja pored sadašnjosti ima i svoju budućnost. 


\addcontentsline{toc}{section}{Literatura}
\appendix
\bibliography{Bibliografija} 
\bibliographystyle{plain}

\end{document}
