

 % !TEX encoding = UTF-8 Unicode

\documentclass[a4paper]{report}

\usepackage[T2A]{fontenc} % enable Cyrillic fonts
\usepackage[utf8x,utf8]{inputenc} % make weird characters work
\usepackage[serbian]{babel}
%\usepackage[english,serbianc]{babel}
\usepackage{amssymb}

\usepackage{color}
\usepackage{url}
\usepackage[unicode]{hyperref}
\hypersetup{colorlinks,citecolor=green,filecolor=green,linkcolor=blue,urlcolor=blue}

\newcommand{\odgovor}[1]{\textcolor{blue}{#1}}

\begin{document}

\title{Dopunite naslov svoga rada\\ \small{Dopunite autore rada}}

\maketitle

\tableofcontents

\chapter{Uputstva}
\emph{Prilikom predavanja odgovora na recenziju, obrišite ovo poglavlje.}

Neophodno je odgovoriti na sve zamerke koje su navedene u okviru recenzija. Svaki odgovor pišete u okviru okruženja \verb"\odgovor", \odgovor{kako bi vaši odgovori bili lakše uočljivi.} 
\begin{enumerate}

\item Odgovor treba da sadrži na koji način ste izmenili rad da bi adresirali problem koji je recenzent naveo. Na primer, to može biti neka dodata rečenica ili dodat pasus. Ukoliko je u pitanju kraći tekst onda ga možete navesti direktno u ovom dokumentu, ukoliko je u pitanju duži tekst, onda navedete samo na kojoj strani i gde tačno se taj novi tekst nalazi. Ukoliko je izmenjeno ime nekog poglavlja, navedite na koji način je izmenjeno, i slično, u zavisnosti od izmena koje ste napravili. 

\item Ukoliko ništa niste izmenili povodom neke zamerke, detaljno obrazložite zašto zahtev recenzenta nije uvažen.

\item Ukoliko ste napravili i neke izmene koje recenzenti nisu tražili, njih navedite u poslednjem poglavlju tj u poglavlju Dodatne izmene.
\end{enumerate}

Za svakog recenzenta dodajte ocenu od 1 do 5 koja označava koliko vam je recenzija bila korisna, odnosno koliko vam je pomogla da unapredite rad. Ocena 1 označava da vam recenzija nije bila korisna, ocena 5 označava da vam je recenzija bila veoma korisna. 

NAPOMENA: Recenzije ce biti ocenjene nezavisno od vaših ocena. Na osnovu recenzije ja znam da li je ona korisna ili ne, pa na taj način vama idu negativni poeni ukoliko kažete da je korisno nešto što nije korisno. Vašim kolegama šteti da kažete da im je recenzija korisna jer će misliti da su je dobro uradili, iako to zapravo nisu. Isto važi i na drugu stranu, tj nemojte reći da nije korisno ono što jeste korisno. Prema tome, trudite se da budete objektivni. 
\chapter{Recenzent \odgovor{--- ocena:} }


\section{O čemu rad govori?}
% Напишете један кратак пасус у којим ћете својим речима препричати суштину рада (и тиме показати да сте рад пажљиво прочитали и разумели). Обим од 200 до 400 карактера.

Apstraktna interpretacija predstavlja tehniku za automatsku statičku analizu koda koja omogućava da otkrijemo greške bez iscrpnog rešavanja izraza i izvršavanja naredbi. Iako je pojam usko povezan sa svojom svrhom u računarstvu, on se može objasniti i na neračunarskim primerima, što je i učinjeno u tekstu.
\section{Krupne primedbe i sugestije}
% Напишете своја запажања и конструктивне идеје шта у раду недостаје и шта би требало да се промени-измени-дода-одузме да би рад био квалитетнији.
Nisam uočio krupnije nedostatke u radu.
\section{Sitne primedbe}
% Напишете своја запажања на тему штампарских-стилских-језичких грешки

\begin{itemize}
  \item Str. 2, pasus 1, red 1: umesto "preformansi", treba "performansi";  \\
  \odgovor{Slovna greška je ispravljena.}  
  
  \item Str. 2, pasus 3, red 5: isto; \\
  \odgovor{Slovna greška je ispravljena.}
  
  \item Str. 2, paus 4, red 2: umesto "... da ponašanje...", treba "... da je ponašanje..."; \\
  \odgovor{Popravljena je greška.}
  
  \item Str. 2, pasus 4, red 6: umesto "optimitacije", treba "optimizacije"; \\
  \odgovor{Slovna greška je ispravljena.}
  
  \item Str. 2, pasus 5, red 6: umesto "'delenje", treba "'deljenje"; \\
  \odgovor{Slovna greška je ispravljena.}
  
  
  \item Str. 2, pasus 5, red 7: umesto "promenjljivih", treba "promenljivih"; isto i na str. 4, pasus 7, redovi 2 i 3; \\
  \odgovor{Slovna greška je ispravljena u svim pojavljivanjima.}
  
  \item Str. 3, pasus 4, red 1: umesto "bi smo", treba "bismo";\\
  \odgovor{Slovna greška je ispravljena.}
  
  \item Str. 5, pasus 5, red 7: umesto "kontantama", treba "konstantama";\\
  \odgovor{Slovna greška je ispravljena.}
  
  \item Str. 4, alinea 1, red 4: umesto "'zatvorsenja"', treba "'zatvorenja"';\\
  \odgovor{Slovna greška je ispravljena.}
  
  \item Str. 4, alinea 4, redovi 3 i 4: umesto "korišćenje analize delova...", treba "korišćenjem analize delova...";\\
  \odgovor{Greška je ispravljena.}
  
  \item Str. 5, poslednja alinea: umesto "privermene", treba "privremene";
  \item Str. 7, pasus 2, red 4: umesto "odznačava", treba "označava";
  \item Str. 7, pasus 5, red 1: umesto "opeatori", treba "operatori";
  \item Str. 7, pretposlednji pasus, red 3: brisati ponovljenu reč "bismo";
  \item Str. 7, pretposlednji pasus, red 5: umesto "'zali"', treba "'znali"';
  \item Str. 7, pretposlednji pasus, red 7: umesto "funckije", treba "funkcije";
  \item Str. 7, poslednji pasus, red 1 i 2: umesto "funckija", treba "funkcija";
  \item Str. 10, pasus 1, red 3: umesto "pokazat da se valjanost očuvava", treba "pokazati da se valjanost čuva";
  \item Str. 11, pasus 1, poslednji red: umesto "konačnoj broju", treba "konačnom broju";
\end{itemize}



\section{Provera sadržajnosti i forme seminarskog rada}
% Oдговорите на следећа питања --- уз сваки одговор дати и образложење

\begin{enumerate}
\item Da li rad dobro odgovara na zadatu temu?\\
Rad odgovara veoma dobro na zahteve postavljene temom.
\item Da li je nešto važno propušteno?\\
Nisam naišao na veće propuste.
\item Da li ima suštinskih grešaka i propusta?\\
Rad nema suštinskih grešaka i propusta. Naišao sam uglavnom na štamparske greške.
\item Da li je naslov rada dobro izabran?\\
Naslov rada jeste dobro izabran zato što se zahteva da se opiše apstraktna interpretacija u celovitosti.
\item Da li sažetak sadrži prave podatke o radu?\\
Sažetak vrlo dobro opisuje suštinu rada kao i teme o kojima se u njemu može saznati više. Daje jasnu sliku o tome šta se može očekivati u radu.
\item Da li je rad lak-težak za čitanje?\\
Rad sam pročitao sa lakoćom. Logičnim sledom uvode se pojmovi koji se dalje koriste u tekstu što omogućava potpuno razumevanje pročitanog.
\item Da li je za razumevanje teksta potrebno predznanje i u kolikoj meri?\\
Potrebno je opšte poznavanje pojmova koji se koriste u programiranju. Sam rad na adekvatan način uvodi nove pojmove tako da ga čini u potpunosti razumljivim i za one manje upućene u tematiku.
\item Da li je u radu navedena odgovarajuća literatura?\\
Literatura koja je navedena odgovara zahtevima teme.
\item Da li su u radu reference korektno navedene?\\
Reference su korektno navođene. Kod naslova [1],[2],[3] i [8] nedostaju godine izdanja.
\item Da li je struktura rada adekvatna?\\
Struktura rada direktno odgovara na postavljena pitanja, te nema potrebe za bilo kakvim prestruktuiranjem niti uvođenjem novih celina.
\item Da li rad sadrži sve elemente propisane uslovom seminarskog rada (slike, tabele, broj strana...)?\\
Rad ne sadrži tabelarne prikaze. Svi ostali uslovi su zadovoljeni.
\item Da li su slike i tabele funkcionalne i adekvatne?\\
Slike koje se koriste na adekvatan način prate i upotpunjuju sadržaj teksta.
\end{enumerate}

\section{Ocenite sebe}
% Napišite koliko ste upućeni u oblast koju recenzirate: 
% a) ekspert u datoj oblasti
% b) veoma upućeni u oblast
% c) srednje upućeni
% d) malo upućeni 
% e) skoro neupućeni
% f) potpuno neupućeni
% Obrazložite svoju odluku

Smatram da sam skoro neupućen u temu rada. Najbliža tema sa kojom sam se susretao bila je semantička analiza u sklopu prevođenja programskih jezika.

\chapter{Recenzent \odgovor{--- ocena:} }


\section{O čemu rad govori?}
% Напишете један кратак пасус у којим ћете својим речима препричати суштину рада (и тиме показати да сте рад пажљиво прочитали и разумели). Обим од 200 до 400 карактера.
Rad ukratko opisuje pojam \textit{apstraktne interpretacije} i na neformalnim primerima prikazuje šta to zapravo predstavlja. Onda je dat primer programskog koda koji treba da približi pojam čitaocu. Na kraju je opisana matematička formalizacija problema koja bi trebalo da služi za dokazivanje korektnosti primene apstraktne interpretacije.


\section{Krupne primedbe i sugestije}
\label{sec: primedbe}
% Напишете своја запажања и конструктивне идеје шта у раду недостаје и шта би требало да се промени-измени-дода-одузме да би рад био квалитетнији.
Tema je \textit{Apstraktna iterpretacija u modernim kompajlerima}, a rad govori samo o apstraktnoj interpretaciji i jedva da pominje predovioce na jednom mestu. Od tri pitanja, koja su postavljena kako bi istraživanje literature i samo pisanje rada bilo olakšano, odgovoreno je na prvo u celosti, i delimično na drugo - potrebno je još primera. Odgovor na treće pitanje (Koje tehnike apstraktne interpretacije se koriste u kompajlerima?) bilo bi dovoljno unapređenje rada kako bi više odgovarao temi.
\\
\\
Poslednji pasus uvoda kao i čitav deo 2.1 je čist prevod dela teksta iz [4] (S. Abramsky and C. Hankin An introduction to abstract interpretation). U potpunosti su preuzeti primeri, kao i njhovo objašnjenje. U uvodu barem postoji citiranje, međutim, u delu 2.1 toga nema.
\\
  \odgovor{Dodana je reference ka literaturi iz koje je preuzet tekst.}
\\
\\
U sekciji 3 postoji identičan problem kao u delu 2.1. Niame, čitava sekcija je prevod teksta iz [10] (Mozilla wiki. Abstract Interpretation). Postoji citiranje na samom kraju čitave sekcije, ali potrebno je opisati primere svojim rečima ili, još bolje, smisliti nove primere pa njih opisati. Ovako je samo prevod teksta i to nije u redu. 
\\
\\
Sekcija 4 uvodi nekoliko definicija i teorema u cilju pokazivanja valjanosti upotrebe ove metode. Međutim, nije mi jasno kako su sve te definicije i teoreme u vezi sa apstraktnom intepretacijom. Potrebno je dodatno objašnjenje kako bi čitaoc zaista bio ubeđen da navedeni formalizam dokazuje valjanost metode. Ova sekcija mogla bi da se zameni još nekim primerima i odgovorom na treće pitanje jer, po mom mišljenju, nije mnogo bitna za temu, a postoje pitanja koja nisu obrađena. 


\section{Sitne primedbe}
% Напишете своја запажања на тему штампарских-стилских-језичких грешки
Pre svega, prevod sa engleskog nije najbolji. Neke rečenice su bukvalno prevođene i nemaju mnogo smisla. Ima dosta dugačkih rečenica koje otežavaju čitanje. Sigurno može sve lepše da se zapiše. Takođe, citiranje je lepše unutar rečenice, odnosno pre tačke na kraju rečenice. Mislim da je na skoro svakom mestu stavljeno nakon tačke i to treba ispraviti.
\\
\odgovor{Sporne rečenice su ponovo prevedene. Dugačke rečenice su razdvojene na više rečenica. Citiranje je stavljeno unutar rečenica.}
\\
\\
U sažetku napisano je \textit{‚‚...jedan neformalan neprogramski primer''}. Nema potrebe da piše i neformalan i neprogramski.
\\
\\
Reč \textit{‚‚promenjljivih''} u sekciji 2 treba ispraviti.
\\
 \odgovor{Slovna greška je ispravljena.}
\\
\\
Deo 2.1 je nejasan. Pre svega, potrebno je preformulisati pasus koji uvodi formu $(1)$ i $(2)$. Možda bi bilo zgodno da se rečenica koja se tiče tih izraza podeli na dve - u tom slučaju bi se prva odnosila na $(1)$, a druga na $(2)$. Ovako je predugačka rečenica i nema mnogo smisla. Takođe, u tom pasusu piše \textit{‚‚Da bismo razumeli apstraktnu interpolaciju...''} - pretpostavljam da je greška u kucanju. U formuli $(2)$, poslednja jednakost ima grešku - stoji znak $-$ umesto $a_-$. Upitna rečenica u pretposlednjem pasusu završava se tačkom umesto upitnikom! Prva rečenica poslednjeg pasusa ove sekcije mi nije jasna. Može se reći da je nespretno prevedeno. 
\\
 \odgovor{Pasus je preformulisan. Greška u kucanju je ispravljena. Slovna greška je ispravljena. Upitna rečenica se sada završava upitnikom. Preformulisana prva rečenica poslednjeg pasusa.} 
\\
\\
Za deo 2.2 imam nekoliko sugestija. Prva rečenica bi lepše zvučala ako bi umesto sa \textit{‚‚Kako''} počinjala sa \textit{‚‚Zbog čega''}. Drugo, ova sekcija uvodi nekoliko novih pojmova, kao što su \textit{stroga analiza} i \textit{analiza moda}. Dobro bi bilo da su zapisane iskošenim slovima i da se dodaju reference za te pojmove. Takođe, prilikom uvođenja novih pojmova, navedeni su nazivi na engleskom u zagradi, što je u redu, ali negde pise\textit{‚‚eng.''} a negde \textit{‚‚engl.''}. Potrebno je odabrati jednu od dve skraćenice kako bi se održala konzistentnost. Poželjno je pisati \textit{‚‚engl.''} jer više odgovara pravilima srpskog jezika. 
\\
\odgovor{Dodan je predlog. Reči su zakošene i dodane su im reference. Promenjena su sva pojavljivanja eng. sa engl.}
\\
\\
Deo 3.1 je u redu po sadržaju, samo bi trebalo napisati \textit{‚‚Terminologije:''} posle slike (savet: promeniti opciju za sliku [h!] u [H], onda će se slika pozicionirati tamo gde je postavite). Umesto \textit{‚‚ivice''}, lepše je reći \textit{‚‚grane''} grafa.
\\
\\
U delu 3.3 postoji pasus-rečenica: \textit{‚‚Ovo izgleda poprilično isto kao i konkretan primer, samo što su sada neke vrednosti apstrahovane, NN i ?, koje predstavljaju skupove konkretnih vrednosti.''}. Možda je greškom stavljen novi red, pošto se naredna rečenica nadovezuje na ovu. Potrebno je preformulisati ovu rečenicu. U narednoj rečenici fali slovo \textit{r} u reči \textit{opeatori}. U pretposlednjoj rečenici pasusa piše: \textit{‚‚tako da 2 + ? -> ?''} - umesto -> koristiti $\longrightarrow$ ($\backslash$longrightarrow). U pasusu nakon primera, u trećoj rečenici, dva puta je napisana reč \textit{bismo}. U narednoj piše \textit{zali} umesto \textit{znali}.
\\
\\ 
Poslednja rečenica u pasusu pred deo 4.1 može da se preformuliše. Umesto \textit{‚‚...i pokazat da se valjanost očuvava...''}, može da piše \textit{‚‚i pokazati da će valjanost biti očuvana''}.
\\
\\
Zaključak je u redu. Poslednja rečenica počinje sa \textit{‚‚Jer''} i to mi se ne dopada. Može nekako da se ukombinuje sa prethodnom rečenicom i da se od njih naprave nove dve rečenice, kako ne bi bila jedna predugačka.

\section{Provera sadržajnosti i forme seminarskog rada}
% Oдговорите на следећа питања --- уз сваки одговор дати и образложење

\begin{enumerate}
\item Da li rad dobro odgovara na zadatu temu?\\
	\noindent Ne u potpunosti. Rad govori o apstraktnoj interpretaciji uopšteno, ali nema govora o njenoj primeni u modernim kompajlerima.

\item Da li je nešto važno propušteno?\\
	\noindent Najveća primedba je to što nije dovoljno pažnje posvećeno kompajlerima.

\item Da li ima suštinskih grešaka i propusta?\\
	\noindent Moglo bi se reći da je suštinska greška to što nije lepo objašnjeno kako se primenjuje u kompajlerima.

\item Da li je naslov rada dobro izabran?\\
	\noindent Smatram da nije. Naslov odgovara sadržaju rada, međutim, sadržaj rada ne odgovara sasvim zadatoj temi, samim tim, naslov ne odgovara temi. Takođe, apstraktna interpretacija je preširok pojam da bi bio naslov rada od 12 strana.

\item Da li sažetak sadrži prave podatke o radu?\\
	\noindent Sažetak nije sasvim u redu. Prva primedba odnosi se na kraj prve rečenice: \textit{‚‚...navedene njene primene u savremenom računarstvu za optimizaciju i verifikaciju softvera.''}. Toga nema u radu. Naveden je primer za propagaciju konstanti što jeste jedan vid optimizacije, ali to nije dovoljno. Još jednu primedbu imam na poslednju rečenicu koja kaže sledeće: \textit{‚‚Na kraju je data i matematička formalizacija koja ukazuje na valjanost upotrebe ove metode za pouzdanu verifikaciju softvera''.} Kao što je već napisano u sekciji \ref{sec: primedbe}, nije jasna veza sa apstraktnom interpretacijom.

\item Da li je rad lak-težak za čitanje?\\
	\noindent Ne može se reći da je lak, jer ima dosta dugačkih rečenica koje se mogu preformulisati u dve, možda i u tri. Sekcija 4 je veoma teška za čitanje. 

\item Da li je za razumevanje teksta potrebno predznanje i u kolikoj meri?\\
	\noindent Rad je potkrepljen primerima i nije potrebno neko posebno predznanje. Dovoljno je osnovno poznavanje programskih jezika kako bi se primeri razumeli. Za sekciju 4 je potrebno predznanje iz algebre kako bi bile jasnije navedene definicije i teoreme, kao i korišćene oznake.

\item Da li je u radu navedena odgovarajuća literatura?\\
	\noindent Literatura je sasvim u redu.

\item Da li su u radu reference korektno navedene?\\
	\noindent Referenca [7] bi trebalo da se odnosi na \textit{In-place update analysis}, međutim, to nije pronađeno. Možda imam pogrešnu verziju navedenog rada, zato predlažem da se ubaci link koji vodi ka ispravnoj verziji, koja zaista sadrži nešto o tome. Ostale reference su u redu.

\item Da li je struktura rada adekvatna?\\
	\noindent Struktura rada je uglavnom u redu. Na nekoliko mesta postoje pasusi-rečenice i potrebno je preformulisati ih i stopiti sa još nekim pasusom ili proširiti još nekom rečenicom.

\item Da li rad sadrži sve elemente propisane uslovom seminarskog rada (slike, tabele, broj strana...)?\\
	\noindent Radu fali tabela, ostale uslove zadovoljava.

\item Da li su slike i tabele funkcionalne i adekvatne?\\
	\noindent Slike koje postoje u radu su adekvatne.

\end{enumerate}

\section{Ocenite sebe}
% Napišite koliko ste upućeni u oblast koju recenzirate: 
% a) ekspert u datoj oblasti
% b) veoma upućeni u oblast
% c) srednje upućeni
% d) malo upućeni 
% e) skoro neupućeni
% f) potpuno neupućeni
% Obrazložite svoju odluku

Moram odabrati odgovor f) potpuno neupućeni. Ova tema nije obrađivana u okviru nastave na fakultetu, a oblast ne spada u domen mog interesovanja.

\chapter{Recenzent \odgovor{--- ocena:} }


\section{O čemu rad govori?}
% Напишете један кратак пасус у којим ћете својим речима препричати суштину рада (и тиме показати да сте рад пажљиво прочитали и разумели). Обим од 200 до 400 карактера.
Rad govori o metodu apstraktne interpretacije pri optimizaciji programskog koda. Objašnjava motivaciju i suštinu ovakvog metoda optimizacije i verifikacije softvera. 
Kroz jednostavan primer koda koji sadrži grananje pokazuje kako možemo izbeći nepotrebne operacije, i umesto toga promenljive zameniti konstantama. 
U poslednjem poglavlju daju objašnjenje kroz matematičku formalizaciju metode.
\section{Krupne primedbe i sugestije}
% Напишете своја запажања и конструктивне идеје шта у раду недостаје и шта би требало да се промени-измени-дода-одузме да би рад био квалитетнији.
Rad bi bio dosta razumljiviji kada bi se primer korišćen u glavi 3 koristio pri obrazloženju matematičke formalizacije u glavi 4. Bez konkretnog primera,
čitaoc je primoran da ''mentalnim mapiranjem''  prevede dat matematički formalizam u konkretan problem, što značajno otežava razumevanje.
Dodatno, svrha sekcije 2.2 je nejasna. Kratak opis nekih statičkih optimizacija ne doprinosi značajno razumevanju same apstraktne interpretacije. 
Dovoljno je bilo navesti samo one koje se koriste u konkretnom primeru u trenutku korišćenja, uz eventualnu referencu gde čitalac može naći neke dodatne metode. 


\section{Sitne primedbe}
% Напишете своја запажања на тему штампарских-стилских-језичких грешки
Rad obiluje jezičkim konstrukcijama koje ne odgovaraju akademskom stilu pisanja. Takođe, na više mesta ima nedovoljno preciznih opisa i stranih reči koje nisu prevedene. 
\par
Primeri:
\begin{itemize}
\item Današnji kućni računari su jači... - \emph{Šta znači ''jači''?}
\\
\odgovor{Dodana je kvalifikacija ispred reči.}

\item Kako je arhitektura postajala sve više i više kompleksna... - \emph{Kako je arhitektura postajala kompleksnija...}
\\
\odgovor{Predlog je prihvaćen.}

\item Apstraktna interpretacija omogućava da otkrijemo runtime greške - \emph{Greške nastale tokom izvršavanja programa?}
\\
\odgovor{Predlog je prihvaćen.}

\item Drugi primer, malo više formalan... - \emph{Reč ''malo'' je ovde suvišna. }
\\
\odgovor{Reč malo je izbačena.}

\item Bez da izvodimo množenje pa odredujemo znak mi... - \emph {Rečenica ne može da počne konstukcijom ''Bez da''. Gramatički neispravno. Ispravno: ''Bez izvođenja operacije množenja...''}
\\
\odgovor{Rečenica je promenjena.}

\item Formatiranje formule (2) nekonzistentno.
\\
\odgovor{Formula je konzistentno formatirana.}

\item Analiza relevantnih klauza će nas terati da komuniciramo sa supersetom - \emph{Nadskupom. }
\\
\odgovor{Predlog je prihvaćen.}

\item ...
\end{itemize}  

\section{Provera sadržajnosti i forme seminarskog rada}
% Oдговорите на следећа питања --- уз сваки одговор дати и образложење

\begin{enumerate}
\item Da li rad dobro odgovara na zadatu temu?\\
Rad delimično odgovara na temu. Motivacija za korišćenje je pojašnjena, a sama suština metode približena kroz (jedan) ilustrativni primer. 
Međutim, radu nedostaje više ilustrativnih primera. Previše pažnje posvećeno matematičkoj formalizaciji, koja nije bila deo zahteva seminraskog rada.
\item Da li je nešto važno propušteno?\\
Ništa od suštinski važnih tema i pitanja nije preskočeno, ali nekima nije posvećeno dovoljno pažnje. Primer: ''Koje tehnike se koriste u kompajlerima''. - Ovom pitanju je imalo smisla posvetiti više pažnje.
\item Da li ima suštinskih grešaka i propusta?\\
Ne. Rad odgovara na sva postavljena pitanja sa manje ili više detalja. 
\item Da li je naslov rada dobro izabran?\\
Isuviše je generički. Teško je zaključiti kojim aspektom \emph{''Apstraktne interpretacije''} se rad tačno bavi na osnovu naslova.
\item Da li sažetak sadrži prave podatke o radu?\\
Da. Dodatno, opis sadržaja je sastavljen tako da redom opisuje obrađene teme i primere.
\item Da li je rad lak-težak za čitanje?\\
Deo sa primerom optimizacije C++ koda je razumljiv i lagan za čitanje i razumevanje, dok je deo sa matematičkom formalizacijom teže razumljiv, jer mu nedostaje ilustrativan primer. Takođe, neformalan govor i stilske greške u radu kvare opšti utisak i otežavaju čitljivost.
\item Da li je za razumevanje teksta potrebno predznanje i u kolikoj meri?\\
Potrebno je razumeti sam proces prevođenja programa. Dodatno, poželjno je imati i određenog predznanja iz Algebre kako bi čitalac mogao da razume opis matematičke formalizacije.
\item Da li je u radu navedena odgovarajuća literatura?\\
Sva literatura deluje adekvatno.
\item Da li su u radu reference korektno navedene?\\
Bez primedbi.
\item Da li je struktura rada adekvatna?\\
Bez primedbi.
\item Da li rad sadrži sve elemente propisane uslovom seminarskog rada (slike, tabele, broj strana...)?\\
Rad ne sadrži ni jednu tabelu.
\item Da li su slike i tabele funkcionalne i adekvatne?\\
Slike su adekvatne i referišu se u tekstu na adekvatan način. 
\end{enumerate}

\section{Ocenite sebe}
% Napišite koliko ste upućeni u oblast koju recenzirate: 
% a) ekspert u datoj oblasti
% b) veoma upućeni u oblast
% c) srednje upućeni
d) malo upućeni 
% e) skoro neupućeni
% f) potpuno neupućeni
% Obrazložite svoju odluku


\chapter{Recenzent \odgovor{--- ocena:} }


\section{O čemu rad govori?}
% Напишете један кратак пасус у којим ћете својим речима препричати суштину рада (и тиме показати да сте рад пажљиво прочитали и разумели). Обим од 200 до 400 карактера.
Rad govori o tehnici apstraktne interpretacije iz tri ugla: neformalnog, formalnog i tehničkog, formirajući objašnjenje kroz nekoliko primera. Predstavljen je problem apstraktne interpretacije i njegova primena u računarstvu.
 
\section{Krupne primedbe i sugestije}
% Напишете своја запажања и конструктивне идеје шта у раду недостаје и шта би требало да се промени-измени-дода-одузме да би рад био квалитетнији.

\textbf{Naslov i tema.} Bitno je naglasiti, ako već ne u temi, barem u apstraktu o kakvoj apstraktnoj interpretaciji se radi. Konstantno se govori da se nešto apstraktno interpretira, ali nije jasno šta.\\ 
\textbf{Izražavanje.} Bilo bi dobro da se obrati pažnja na način izražavanja u radu. Tekst je teži za čitanje zbog načina izražavanja, nego što bi to trebalo da bude slučaj. Izričito sugerišem kolegama da, kroz barem još jednu iteraciju, prođu kroz tekst.\\ 
\textbf{Prevod.} Prevod je, često, suviše bukvalan i doslovan sa engleskog jezika, čiji se uticaj preterano oseća kroz tekst. Trebalo bi se posebno osvrnuti na to da se prevod prilagodi srpskom jeziku.\\
\textbf{Reference i objašnjenja pojmova.} U sekciji 2, u prvom pasusu, fali objašnjenje na delu vezanom za sigurne informacije. U sekciji 2, delu 1, fali objašnjenje za apstraktnu interpolaciju. U 2.2 fali objašnjenje za arhitekturu 5. generacije (navesti primere i ubaciti kratko objašnjenje).\\
\odgovor{Objašnjenje je dodano. Apstraktna interpolacija je štamparska greška i ispravljena je. Pasus je izbačen.}
\\
\textbf{Potencijalni plagijarizam.} Sugeriše se kolegama da obrate pažnju na preuzimanje teksta iz literature. Potrebno je da se, kao što je i gore navedeno, više pažnje obrati na izražajnost prilikom izlaganja rada, a time i na to da se na bolji način predstavi tema.

\section{Sitne primedbe}
% Напишете своја запажања на тему штампарских-стилских-језичких грешки
Mejlove treba navesti da drugačiji način, štrče iz forme.
Da bi rad bio bolji trebalo bi koristiti različite vidove naglašavanja teksta, kao što su: bold, italic, i ostali. Za davanje primera, trebalo je koristiti odgovarajuću formu za primer.\\

\subsection{Uvod}
Deo gde se prvi put pominje teraflop treba proširiti objašnjenjem šta je teraflop. Iz same reči teraflop ne vidi se da je ovo naziv na engleskom $"$(eng. floating point operations per second)$"$.\\
\odgovor{U samom radu od početka stoji iza reči teraflops u zagradama pun naziv reči.}
\\
$"$Glavni krivci za ovakvo poboljšanje u brzini računara su dva aspekta.$"$
Glavne krivce za ovakav porast brzine računara pronalazimo kroz dva aspekta. Zameniti $"$sve više i više kompleksna $"$ sa $"$ sve kompleksnija $"$ ili sličnim. Prepraviti reč $"$optimitacije$"$. \\
\odgovor{Reči su zamenjene.}

\subsection{Apstraktna interpretacija}
$"$Kao što se vidi iz prethodnog poglavlja apstraktna interpretacija je
tehnika za automatsku statičku analizu.$"$ i dalje nije jasno šta analiziramo. Fali objašnjenje šta su sigurne informacije. Prepraviti reč $"$delenje$"$. Dodati zareze oko reči takođe gde god da je to potrebno. Prepraviti reč $"$promenjljivih$"$. \\
\odgovor{Dodano objašnjenje za analizu. Dodano objasnjenje sigurnih informacija. Slovne greške ispravljene.}

\subsubsection{Problem koji se rešava}
Apoziciju $"$malo više formalan$"$ prepraviti u $"$formalniji$"$,a $"$se gradi$"$ predstaviti u obliku $"$gradi se$"$. Fali zarez pre $"$ali$"$ u $"$korektnost interpretacije ali treba$"$. Obratiti pažnju na ovaj zarez i u daljem tekstu. Fali zarez pre $"$a$"$ u $"$Ako bi stavili znak (0, +, -) a$"$. \\
\odgovor{Prepravljene su reči i dodani zarezi.}

\subsubsection{Korišćenje u računarstvu}
Prepraviti reč $"$zatvorsenja$"$. Obratiti pažnju da se ne kaže $"$promenjljiva$"$ već $"$promenljiva$"$. \\
\odgovor{Slovne greške ispravljene.}

\subsection{Primeri}
Ispraviti reč $"$vrednost$"$ u vrednosti.Ispraviti reč $"$kontantama$"$. 
\subsubsection{Grafovi kontrole toka}
Ispraviti reč $"$Privermene$"$.

\subsection{Zaključak}
Zaključak je dobro napisan. Obratiti pažnju na nepravilno razdvajanje rečenica. (Na primer, rečenicu ne počinjati sa $"$jer$"$.)
\section{Provera sadržajnosti i forme seminarskog rada}
% Oдговорите на следећа питања --- уз сваки одговор дати и образложење

\begin{enumerate}
\item Da li rad dobro odgovara na zadatu temu?\\Uglavnom da, ali bi bilo dobro neke delove popraviti.
\item Da li je nešto važno propušteno?\\ Propušteni su primeri, u zahtevima je 5 primera u radu ima manje.
\item Da li ima suštinskih grešaka i propusta?\\ Nema, evenutalno 4. sekcija bi se možda mogla zameniti nekim razumljivijim i manje komplikovanim delom. Nekako izgleda kao da joj tu nije mesto u odnosu na ostatak rada.
\item Da li je naslov rada dobro izabran?\\ Donekle jeste, međutim, nije jasno iz naslova o čemu se radi, odnosno, nije jasno šta se apstraktno interpretira.
\item Da li sažetak sadrži prave podatke o radu?\\ Da.
\item Da li je rad lak-težak za čitanje?\\Izuzetno težak za čitanje, najviše zbog doslovnog prevoda i loše izražajnosti.
\item Da li je za razumevanje teksta potrebno predznanje i u kolikoj meri?\\ Potrebno je određeno, više nego osnovno, predznanje. Uz dodatke referenci i objašnjenja određenih pojmova, taj nivo bi mogao da se smanji bez previše poteškoća po autore.
\item Da li je u radu navedena odgovarajuća literatura?\\ Da.
\item Da li su u radu reference korektno navedene?\\Jesu.
\item Da li je struktura rada adekvatna?\\Jeste, ali nema tabele.
\item Da li rad sadrži sve elemente propisane uslovom seminarskog rada (slike, tabele, broj strana...)?\\ Rad ispunjava dobar broj strana, ali i minimalni broj stavki za literaturu. Sugestija je da kolege eventualno pogledaju još po neki izvor. Tabele nema. Rad sadrži dve slike, što je u redu. 
\item Da li su slike i tabele funkcionalne i adekvatne?\\ Nema tabele, slika ima i u redu su.
\end{enumerate}

\section{Ocenite sebe}
% Napišite koliko ste upućeni u oblast koju recenzirate: 
% a) ekspert u datoj oblasti
% b) veoma upućeni u oblast
% c) srednje upućeni
% d) malo upućeni 
% e) skoro neupućeni
% f) potpuno neupućeni
% Obrazložite svoju odluku
Moja upućenost u oblast je između skoro neupućen i malo upućen.

\chapter{Recenzent \odgovor{--- ocena:} }


\section{O čemu rad govori?}
% Напишете један кратак пасус у којим ћете својим речима препричати суштину рада (и тиме показати да сте рад пажљиво прочитали и разумели). Обим од 200 до 400 карактера.
Apstraktna interpretacija nam pomaže u optimizaciji koda. Sa njom se dolazi do automatizovane transformacije samog koda. Smanjuje se opterećenje koje je trenutno na programerima. Analizira se kod i uviđa se koliko je neka promenljiva zavisna od drugih, koliko se njena vrednost menja kroz samo izvršavanje programa. Ovom analizom dolazimo do toga koji deo možemo zameniti apstrakovanjem koda. 

\section{Krupne primedbe i sugestije}
% Напишете своја запажања и конструктивне идеје шта у раду недостаје и шта би требало да се промени-измени-дода-одузме да би рад био квалитетнији.
Trebalo je da se da još primera apstraktne interpretacije, možda umesto formalizacije koja koliko vidim nije bila tražena iako je lep dodatak.\\
U drugoj glavi je rečeno da se posmatra sabiranje, sama oznaka je malo nejasna, da li je podvučeni plus ili se razmatra i oduzimanje ili je to pri transformaciji? Možda bi bilo dobro staviti referencu u tom delu ako fali. 

\section{Sitne primedbe}
% Напишете своја запажања на тему штампарских-стилских-језичких грешки
2.2. Stroga analiza - zatvorsenja piše u pasusu. \\
\odgovor{Slovna greška je ispravljena.}
\\

3.3. Možda ubaciti klarifikaciju zašto 2 na 64ti \\
7 strana - Tačno smo zali 
\section{Provera sadržajnosti i forme seminarskog rada}
% Oдговорите на следећа питања --- уз сваки одговор дати и образложење

\begin{enumerate}
\item Da li rad dobro odgovara na zadatu temu?\\
U mejlu koji im je poslat se traži  da se prikaže pet odabranih primera, ako sam dobro razumela to pitanje, onda treba dodati u rad još primera.   

\item Da li je nešto važno propušteno?\\
Isti odgovor kao na prošlo pitanje.

\item Da li ima suštinskih grešaka i propusta?\\
Nema.

\item Da li je naslov rada dobro izabran?\\
Da.

\item Da li sažetak sadrži prave podatke o radu?\\
Sadrži prave podatke.

\item Da li je rad lak-težak za čitanje?\\
Lak je za čitanje.

\item Da li je za razumevanje teksta potrebno predznanje i u kolikoj meri?\\
Potrebno je predznanje, pogotovo iz matematike za deo gde se govori o formalizaciji.

\item Da li je u radu navedena odgovarajuća literatura?\\
Jeste.

\item Da li su u radu reference korektno navedene?\\
Korektno su navedene.

\item Da li je struktura rada adekvatna?\\
Jeste.

\item Da li rad sadrži sve elemente propisane uslovom seminarskog rada (slike, tabele, broj strana...)?\\
Nema tabele, ostalo je ispunjeno.

\item Da li su slike i tabele funkcionalne i adekvatne?\\
Slika je adekvatna.
\end{enumerate}

\section{Ocenite sebe}
% Napišite koliko ste upućeni u oblast koju recenzirate: 
% a) ekspert u datoj oblasti
% b) veoma upućeni u oblast
% c) srednje upućeni
% d) malo upućeni 
% e) skoro neupućeni
% f) potpuno neupućeni
% Obrazložite svoju odluku

C, imam određeno predznanje iz kompilatora.

\chapter{Dodatne izmene}
%Ovde navedite ukoliko ima izmena koje ste uradili a koje vam recenzenti nisu tražili. 

\end{document}
