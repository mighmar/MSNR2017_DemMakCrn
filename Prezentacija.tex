\documentclass[xetex,mathserif,serif]{beamer}

\usetheme{Darmstadt}
\usecolortheme{beetle}

\usepackage{graphicx}
\makeatletter
\providecommand{\bigsqcap}{%
  \mathop{%
    \mathpalette\@updown\bigsqcup
  }%
}
\newcommand*{\@updown}[2]{%
  \rotatebox[origin=c]{180}{$\m@th#1#2$}%
}
\makeatother


\title % (optional, only for long titles)
{Apstraktna interpretacija u modernim kompajlerima}
\subtitle{Seminarski rad u okviru kursa Metodologija stručnog i naučnog rada}
\author[Demonja, Maksimović, Crnobrnja] % (optional, for multiple authors)
{O.~Demonja, S.~Maksimović i M.~Crnobrnja}
\institute% (optional)
{
  Matematički fakultet\\
  Univerzitet u beogradu
}
\date % (optional)



\begin{document}
  \frame{\titlepage}
  \begin{frame}
    \frametitle{Formalizacija}
	    \framesubtitle{Skupovi konkretnih i apstraktnih stanja}
		\begin{center}
			\begin{itemize}
				\item $v \in V$ skup konkretnih stanja
				\item $v_{1} \rightsquigarrow v_{2}$ relacija prelaska
				\item $... \rightsquigarrow v_{n} \rightsquigarrow v_{n}$ zaustavljanja programa
				\item $l \in L$ prostor svojstava
				\item $l_{1} \rightarrow l_{2}$ \emph{funkcija} prelaska
				\item $\sqsubseteq \; \; \subset \; L \times L$ relacija poretka nad $L$
				\begin{itemize}
					\item $a \sqsubseteq a$ refleksivnost
					\item $a \sqsubseteq b \wedge b \sqsubseteq a \implies a = b$ antisimetričnost
					\item $a \sqsubseteq b \wedge b \sqsubseteq c \implies a \sqsubseteq c$ tranzitivnost
				\end{itemize}
				\item potpuna mreža
				\begin{itemize}
					\item $\forall l_0 \in L^{\prime} \; \; l_0 \sqsubseteq \bigsqcup_{l \in L^{\prime}} l$
					\item $\forall l_0 (\forall l \in L^{\prime} \; l \sqsubseteq	l_0) \implies \bigsqcup_{l \in L^{\prime}} l \sqsubseteq l_0 $
					\item analogno za $\bigsqcap_{l \in L^{\prime}} l$ ...
				\end{itemize}
			\end{itemize}
		\end{center}
  \end{frame}
  \begin{frame}
    \frametitle{Formalizacija}
	    \framesubtitle{Skupovi konkretnih i apstraktnih stanja}
		\begin{center}
		    \begin{figure}
				\includegraphics[scale=0.5]{Rho.png}
				\end{figure}
		\end{center}
  \end{frame}
  \begin{frame}
    \frametitle{Formalizacija}
    \begin{figure}
		\begin{center}
		\includegraphics[scale=0.5]{Rho_prime.png}
		\end{center}
	\end{figure}
  \end{frame}
  \begin{frame}
    \frametitle{Formalizacija}
    \framesubtitle{Fiksne tačke}
	\begin{center}
		\begin{itemize}
			\item $x = f_{L}(x)$
			\item Čine potpunu mrežu.
		\end{itemize}
		\end{center}
  \end{frame}
% etc
\end{document}